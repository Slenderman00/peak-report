% !TeX root = Report.tex
\documentclass[12pt]{article}

% Package imports (organized and deduplicated)
\usepackage{biblatex}
\usepackage{changepage}
\usepackage{color}
\usepackage{enumitem}
\usepackage{float}
\usepackage{graphicx}
\usepackage{listings}
\usepackage{sectsty}
\usepackage{xcolor}
\usepackage[breaklinks=true]{hyperref}
\usepackage{xurl}
\setcounter{biburlnumpenalty}{100}
\setcounter{biburlucpenalty}{100}
\setcounter{biburllcpenalty}{100}

% Make bibliography ragged right instead of justified
\AtBeginDocument{
  \renewcommand{\bibsetup}{\raggedright}
}
% Document configuration
\restylefloat{table}
\graphicspath{{./images/}}
\addbibresource{Library.bib}
\subsectionfont{\fontsize{12}{14}\selectfont}

% Author information
\author{
    Joar Heimonen\\
    \texttt{contact@joar.me}
}

% Title configuration
\title{
    \textbf{Peak:}\\
    \large Leveraging Proven Technologies to Create\\
    the Distributed System of the Future
}
\date{\today}

\begin{document}
\maketitle

\begin{abstract}
    The current paradigm of cloud computing heavily relies on proxies, which introduce single points of failure in systems meant to be distributed. 
    We propose a radical simplification of the current architecture by leveraging the abundance of IPv6 addresses and utilizing modern purpose-built 
    DNS servers to create a distributed system that is both more reliable and scales drastically better than the current cloud computing paradigm. 
    The proposed system utilizes cluster-level DNS servers that dynamically manage service discoverability using Prometheus for monitoring. 
    Service-to-service communication is handled through JSON Web Tokens, creating an intercommunication system that scales 1:1 with the number of users. 
    The system achieves robust fault tolerance through native DNS client failover capabilities, leveraging the universal support for multiple 
    record resolution and automatic retry behavior present in all modern DNS implementations. 
    This eliminates the need for custom failover logic while providing battle-tested reliability mechanisms that operate transparently to applications. 
    This paper describes the implementation of Peak, a proof-of-concept implementation of the proposed architecture, along with its development process and tooling.
\end{abstract}

\pagebreak

\tableofcontents

\pagebreak


\section{Introduction}
The evolution of distributed web systems can be traced back to early protocols like FastCGI, which introduced the concept of long-running service processes.
This marked a significant departure from CGI's one-process-per-request approach,
and established patterns of intermediary communication that would later become ubiquitous in cloud computing.
FastCGI's architecture, with its process manager mediating between web servers and application processes, was a precursor to the modern cloud computing paradigm,
where services are abstracted into containers and orchestrated by centralized management systems.
\\
\\
The modern cloud computing landscape has evolved this simple concept into a complex ecosystem of proxies,
load balancers, message brokers, and service discovery mechanisms. While this architecture has served us well, it introduces
significant operational complexity and creates single points of failure in systems designed to be distributed.
Current deployments typically rely on multiple layers of proxies for routing, discovery, and load balancing, each representing
a potential point of failure.
\\
\\
With the widespread adoption of IPv6, we have entered a new era in distributed systems architecture. RIPE NCC's allocation policy
provides Local Internet Registries with /29 blocks, each containing over 500,000 /48 networks. This abundance of addresses eliminates
the need for network address translation and, by extension, many of the proxy-based patterns that evolved around address scarcity.
Modern DNS clients support multiple record resolution and automatic retry behavior, providing a robust failover mechanism that is 
transparent to applications.
\\
\\
In this paper we present Peak and PeakDNS, a proof-of-concept implementation of a distributed system that uses cluster-level 
DNS servers to manage service discoverability and load-based record management.


\section{Technical background}
\subsection{Réseaux IP Européens (RIPE)}
Réseaux IP Européens (RIPE)\cite{WelcomeRIPERIPE2024} is a regional internet registry (RIR) 
that allocates and registers IP addresses in Europe, the Middle East, and parts of Central Asia.

\subsection{PeakDNS}
PeakDNS\cite{heimonenPeakDNS2024} is a purpose-built DNS server that manages service discoverability and 
load-based record management for the Peak distributed system.
It is designed to integrate with Prometheus for monitoring and alerting.
and Kubernetes for container discoverability.

\subsection{Domain Name System}
The Domain Name System (DNS)\cite{DomainNamesImplementation1987} is a hierarchical and decentralized naming system for computers, 
services, or other resources connected to the Internet or a private network.
\subsubsection{DNS Record Types}
DNS records are used to provide information about a domain or hostname.
\begin{itemize}
    \item A (Address) - Maps a domain to an IPv4 address.
    \item AAAA (Address) - Maps a domain to an IPv6 address.
    \item CNAME (Canonical Name) - Maps a domain to another domain.
    \item MX (Mail Exchange) - Maps a domain to a mail server.
    \item NS (Name Server) - Maps a domain to a name server.
    \item PTR (Pointer) - Maps an IP address to a domain.
    \item SOA (Start of Authority) - Provides authoritative information about a DNS zone.
    \item SRV (Service) - Maps a domain to a service.
    \item TXT (Text) - Provides arbitrary text data.
\end{itemize}

\subsection{Records with multiple answers}
DNS clients support multiple record resolution, which allows multiple records to be returned for a single query.
This feature is used to provide fault tolerance and load balancing by returning multiple IP addresses for a single domain.
It is up to the client to decide which record to use, and most modern clients implement automatic retry behavior.

\subsection{IPv6}
Internet Protocol version 6 (IPv6) is the most recent version of the Internet Protocol (IP)
\subsubsection{IPv6 Address Structure}
IPv6 addresses are 128 bits long and are represented as eight groups of four hexadecimal digits separated by colons.
This introduces an IP space that contains $2^{128}$ (approximately 340 undecillion, or $3.4 \times 10^{38}$) addresses, 
eliminating the need for network address translation. 
To put this in perspective, the entire IPv4 address space ($2^{32}$ addresses) could fit into the IPv6 address space $2^{96}$ times 
(approximately 79 octillion times).
In other words, we could replicate the entire Internet's IPv4 address space \\79,228,162,514,264,337,593,543,950,336 times within IPv6's address space.
\subsubsection{IPv6 Address Allocation policy}
RIPE NCC's allocation policy provides Local Internet Registries with /32 up to /29 blocks, each containing over 500,000 /48 networks.
To qualify for these allocations the Local Internet registry 
"must have a plan for making sub-allocations to other organizations and/or End Site assignments within two years."\cite{IPv6AddressAllocation}.
This makes large IPv6 allocations available to any organization that can demonstrate a need for them.

\subsection{JSON Web Tokens}
JSON Web Tokens (JWT)\cite{jonesJSONWebToken2015} are an open, industry-standard RFC 7519 method for representing claims securely between two parties.

\subsection{Docker}
Docker is a set of platform as a service (PaaS) products that use OS-level virtualization to deliver software in packages called containers.

\subsection{Kubernetes}
Kubernetes is an open-source container-orchestration system for automating computer application deployment, scaling, and management.

\subsection{Prometheus}
Prometheus is an open-source monitoring and alerting toolkit originally built at SoundCloud.

\pagebreak
\printbibliography

\end{document}